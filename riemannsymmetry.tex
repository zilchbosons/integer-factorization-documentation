\section{Deriving the Riemann Symmetry Relations from the Number and Its Reverse}
We write the number and its reverse stacked one above the other. It does not matter which is stackeb above which one. What matters is the relative symmetry between the Riemann Zeros if they exist in the digit patterns of the two vectors.\\
As an example, if N = 4251161764252561,\\
The stack,
4251161764252561\\
1652524671611524\\
In column \#2 and \#3,\\ 
We have,\\
25\\
65

\subsection{Observations}
\begin{enumerate}
\item 25 and 65 are Riemann zeros and hence symmetrical.
\item 52 and 56 are Riemann zeros and hence symmetrical. 
\item 25 and 56 are Riemann zeros and hence anti-symmetrical.
\item 52 and 65 are Riemann zeros and hence anti-symmetrical.
\end{enumerate}



