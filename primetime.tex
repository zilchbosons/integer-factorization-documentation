\section{Prime Time : Counting Primes }

Once the Riemann Zeros and their Relative symmetries have been identified. We follow the simple process for symmetrical and anti-symmetrical Riemann zeros : 
For symmetrical Riemann Zeros, we fetch the integral part of the Riemann Zeros and 10 decimal digits for a total of twelve digits excluding the decimal point.
My point of reference is the \\
\url{http://www.dtc.umn.edu/~Odlyzko/zeta_tables/zeros2}{}.\\
For N = 4251161764252561 \\
Stack is, \\
4251161764252561\\
1652524671611524\\
Symmetrical Zeros : \\
25 and 65\\
52 and 56\\
Asymmetrical Zeros:\\
52 and 65\\
25 and 56\\
For symmetry,\\
including the decimal digits\\
25.010857580 (rounded off to 9 decimal digits) \\
65.112544048 (rounded off to 9 decimal digits) \\
Primes: 11, 02, 47, 05 \\
Number of Primes: 4\\
\\
And \\
\\
52.970321478 (rounded off to 9 decimal digits) \\
56.446247697 (rounded off to 9 decimal digits) \\
Primes: 47, 23, 17, 71, 79, 97 \\
Number of Primes: 6\\
\\
For cross or anti-symmetry,\\
Original Zero:\\
25.010857580\\
65.112544048\\
Primes: 11, 02, 47, 05\\
Number of Primes: 4\\
\\
Cross-Symmetry \#2 :\\
Original Zero:\\
52.970321478\\
56.446247697\\
Primes: 47, 23, 17, 71, 79, 97\\
Number of Primes: 6\\
This completes the anlysis of prime numbers in the retrieved Riemann zeros (upto 10 decimal places). With the Riemann zeros being in the range from 14 to 98.

So to summarize the yield from the Prime Number Counting Step:\\
Iteration 1 (without rotation):\\
\begin{enumerate}
\item From Symmetry: \\
\subitem Number of Primes: 4\\
\subitem Number of Primes: 6\\
\end{enumerate}

\begin{enumerate}
\item From Cross Symmetry:\\
\subitem Number of Primes: 4\\
\subitem Number of Primes: 6\\
\end{enumerate}

